\documentclass[11pt]{scrartcl}
\usepackage[sexy]{../../../evan}
\usepackage{graphicx}
\usepackage{float}
\usepackage{pgfplots}
\usepackage{subcaption}
\usepackage{algpseudocode}
\pgfplotsset{compat=1.18}
\definecolor{dg}{RGB}{2,101,15}
\newtheoremstyle{dotlessP}{}{}{}{}{\color{dg}\bfseries}{}{ }{}
\theoremstyle{dotlessP}
\newtheorem{property}[theorem]{Property}
\newtheoremstyle{dotlessN}{}{}{}{}{\color{teal}\bfseries}{}{ }{}
\theoremstyle{dotlessN}
\newtheorem{notation}[theorem]{Notation}
\newtheoremstyle{dotN}{}{}{}{}{\color{teal}\bfseries}{.}{ }{}
\theoremstyle{dotN}
\newtheorem{solution}{Solution}
% Shortcuts
\DeclarePairedDelimiter\ceil{\lceil}{\rceil} % ceil function
\DeclarePairedDelimiter\flr{\lfloor}{\rfloor} % floor function

\DeclarePairedDelimiter\paren{(}{)} % parenthesis

\newcommand{\df}{\displaystyle\frac} % displaystyle fraction
\newcommand{\qeq}{\overset{?}{=}} % questionable equality

\newcommand{\Mod}[1]{\;\mathrm{mod}\; #1} % modulo operator

\newcommand{\comp}{\circ} % composition

% Text Modifiers
\newcommand{\tbf}{\textbf}
\newcommand{\tit}{\textit}

% Sets
\DeclarePairedDelimiter\set{\{}{\}}
\newcommand{\unite}{\cup}
\newcommand{\inter}{\cap}

\newcommand{\reals}{\mathbb{R}} % real numbers: textbook is Z^+ and 0
\newcommand{\ints}{\mathbb{Z}}
\newcommand{\nats}{\mathbb{N}}
\newcommand{\complex}{\mathbb{C}}
\newcommand{\tots}{\mathbb{Q}}

\newcommand{\degree}{^\circ}

% Counting
\newcommand\perm[2][^n]{\prescript{#1\mkern-2.5mu}{}P_{#2}}
\newcommand\comb[2][^n]{\prescript{#1\mkern-0.5mu}{}C_{#2}}

% Relations
\newcommand{\rel}{\mathcal{R}} % relation

\setlength\parindent{0pt}

% Directed Graphs
\usetikzlibrary{arrows}
\usetikzlibrary{positioning,chains,fit,shapes,calc}

% Contradiction
\newcommand{\contradiction}{{\hbox{%
    \setbox0=\hbox{$\mkern-3mu\times\mkern-3mu$}%
    \setbox1=\hbox to0pt{\hss$\times$\hss}%
    \copy0\raisebox{0.5\wd0}{\copy1}\raisebox{-0.5\wd0}{\box1}\box0
}}}

% CS 
% NP
% Modulo without space
\newcommand{\nmod}[1]{\;\mathrm{mod}\;#1}
\newcommand{\np}{\texttt{NP}_\texttt{search}}
\newcommand{\p}{\texttt{P}_\texttt{search}}
\newcommand{\nph}{\texttt{NP}_\texttt{search}\text{-hard}}
\newcommand{\npc}{\texttt{NP}_\texttt{search}\text{-complete}}
\newcommand{\EXP}{\texttt{EXP}_\texttt{search}}
\newcommand{\xxhash}[2]{\rotatebox[origin=c]{#2}{$#1\parallel$}}

\title{CS 124: Data Structures and Algorithms}
\subtitle{PSet 1}
\author{Cao and Ziv}
\date{\today}
\newcommand{\courseNumber}{CS 124}
\newcommand{\courseName}{Data Structures and Algorithms}
\newcommand{\psetName}{ProgSet 2}
\newcommand{\dueDate}{Due: Wednesday, March 27, 2024}
\newcommand{\name}{Denny Cao and Ossimi Ziv}
\renewcommand{\theques}{\thesection.\alph{ques}} % Change subtheo counter for alpha output
\declaretheorem[style=basehead,name=Answer,sibling=theorem]{ans}
\renewcommand{\theans}{\thesection.\alph{ans}}


%++++++++++++++++++++++++++++++++++++++++
% title stuff
\makeatletter
\renewcommand{\maketitle}{\bgroup\setlength{\parindent}{0pt}
    \begin{flushleft}
        {\Large\textbf{\@title}} \\ \vskip0.2cm
        \begingroup
            \fontsize{12pt}{12pt}\selectfont
            \courseNumber: \courseName 
        \endgroup \vskip0.3cm
        \dueDate \hfill\rlap{}\textbf{\name} \\ \vskip0.1cm
        \hrulefill
    \end{flushleft}\egroup 
}
\makeatother

\title{\psetName}
\begin{document}
\maketitle
\thispagestyle{plain}
\section{Quantitative Results}
\subsection{Analytical Crossover Calculation}
The crossover point is the point at which \texttt{strassen} becomes faster than the naive algorithm. We can calculate the crossover point by setting the two algorithms equal to each other and solving for $n$.
\\

The runtime of \texttt{standard} is
\[
T(n)=n^2(2n-1)
\] 
assuming that all arithmetic operations
have a cost of 1 by Task 1. This is because, for each of the resulting $n^2$ numbers in the resulting matrix, there
are a total of $n$ multiplications and $n-1$ additions.
\\

For \texttt{strassen}, we only run ``one layer'' of the algorithm, with the subproblems using \texttt{standard} in
order to calculate the resulting matrix. There are two cases:
\begin{enumerate}
    \item $n$ is a power of 2. This means that the runtime of \texttt{strassen} is
        \[
T'(n)=7T(n/2)+18(n/2)^2
        \] 
as there are 7 subproblems and 18 matrix additions ($(n/2)^2$ elements in each) in the algorithm, since with powers
of 2, we can continuously halve the matrix and the subproblems will remain powers of 2 (even).
\\

We can now set the two algorithms equal to each other and solve for $n$:
\begin{align*}
    2n^3 - n^2 &= 7(2(n/2)^3 - (n/2)^2) + 18(n/2)^2 \\
     &= \frac{7}{4}n^3 - \frac{7}{4}n^2 + \frac{18}{4}n^2 \\
    0 &= -\frac{1}{4}n^3 + \frac{15}{4}n^2 \\
     &= -\frac{1}{4}n^2(n - 15) \\
     n &= 15
\end{align*}
We can see that the crossover point is $n_0 = 15$.
\item $n$ is not a power of 2. This means that the runtime of \texttt{strassen} is 
\[
T'(n) = 7T((n+1)/2) + 18((n+1)/2)^2
\] 
    This is because the algorithm will pad the matrix by adding a column and row of zeros to the
    matrix, making the input size for subproblems $(n+1)/2$.
    \\

    We can now set the two algorithms equal to each other and solve for $n$:
    \begin{align*}
        2n^3 - n^2 &= 7(2((n+1)/2)^3 - ((n+1)/2)^2) + 18((n+1)/2)^2 \\
               0    &=\frac{7}{4}(n+1)^{3}+\frac{11}{4}(n+1)^{2}-2n^{3}+n^{2} \\
                    &= \frac{7}{4}\left(n^{3}+3n^{2}+3n+1\right)+\frac{11}{4}\left(n^{2}+2n+1\right)-2n^{3}+n^{2}\\
                    &= -\frac{1}{4}n^{3}+9n^{2}+\frac{43}{4}n+\frac{18}{4} \\
               n&= 37.17
           \end{align*}
           Thus, the crossover point is around $n_0 = 37$.
\end{enumerate}
We combine the two cases to get the crossover point for all $n$:
\begin{itemize}
    \item For $n < 15$, \texttt{standard} is faster. 
    \item For $15 \leq n \leq 37$, it is unclear which algorithm is faster.
        \item For $n > 37$, \texttt{strassen} is faster.
\end{itemize}
Thus, the theoretical crossover point is $n_0 = 37$.
\subsection{Empirical Crossover}
We obtain the empirical crossover point by running the two algorithms on random matrices of size $n$ with entires 0
and 1 and timing them. We then
plot the results and find the point at which, for all $n$ greater than the crossover point, \texttt{strassen} is
faster. A subset of the results are shown below, taking the average of 5 runs for each matrix size between 1 and 50.
\begin{figure}[H]
    \begin{subfigure}{0.5\textwidth}
\resizebox{\columnwidth}{!}{%
        \begin{tabular}{c|c|c}
            
       Matrix Size
       ($n$) & Average Time Strassen (ms) & Average Time Standard (ms) \\
\hline
       1 & 0.04005432 & 0.01764297 \\
         2 & 0.25367737 & 0.04434586 \\
            3 & 2.44951248 & 0.24557114 \\
            4 & 1.03759766 & 0.45967102 \\
            5 & 2.76184082 & 0.08940697 \\
            6 & 0.08349419 & 0.05903244 \\
            7 & 0.19207001 & 0.09231567 \\
            8 & 0.14777184 & 0.13208389 \\
            9 & 0.29582977 & 0.18420219 \\
            10 & 0.26364326 & 0.24600029 \\
            11 & 0.44384003 & 0.32444000 \\
            12 & 0.41499138 & 0.42495727 \\
            13 & 0.70180892 & 0.57215691 \\
            14 & 0.67152977 & 0.70323944 \\
            15 & 0.98776817 & 0.87027550 \\
            16 & 0.96721649 & 1.02620125 \\
            17 & 1.36899948 & 1.19962692 \\
            18 & 1.30710602 & 1.41773224 \\
            19 & 1.76682472 & 1.64866447 \\
            20 & 1.73182487 & 1.92041397 \\
            21 & 2.31013298 & 2.22172737 \\
            22 & 2.32915878 & 2.55317688 \\
            23 & 3.03268432 & 2.95033455 \\
            24 & 3.01985741 & 3.30181122 \\
            25 & 3.78847122 & 3.72204781 \\
    \end{tabular}}
    \end{subfigure}
    \begin{subfigure}{0.5\textwidth}
        \resizebox{\columnwidth}{!}{%
        \begin{tabular}{c|c|c}
            
       Matrix Size
       ($n$) & Average Time Strassen (ms) & Average Time Standard (ms) \\
\hline
26 & 3.75418663 & 4.20546532 \\
27 & 4.70714569 & 4.67915535 \\
28 & 4.69846725 & 5.22465706 \\
29 & 5.78403473 & 5.77650070 \\
30 & 5.74755669 & 6.42280579 \\
31 & 6.94327354 & 7.02953339 \\
32 & 6.93907738 & 7.74512291 \\
33 & 8.31937789 & 8.45537186 \\
34 & 8.70699883 & 9.58261489 \\
35 & 9.94524956 & 10.25118828 \\
36 & 9.80730057 & 10.96653938 \\
37 & 11.51275635 & 12.00428009 \\
38 & 11.48490906 & 12.88061142 \\
39 & 13.39626312 & 13.96603584 \\
40 & 13.42787743 & 15.08932114 \\
41 & 15.63568115 & 16.28928185 \\
42 & 15.51976204 & 17.43607521 \\
43 & 17.73638725 & 18.69397163 \\
44 & 17.71345139 & 19.96340752 \\
45 & 20.18704414 & 21.47617340 \\
46 & 20.11113167 & 22.76115417 \\
47 & 22.77207374 & 24.14793968 \\
48 & 22.81498909 & 25.85563660 \\
49 & 25.80103874 & 27.45056152 \\
50 & 27.79173851 & 29.60662842 \\
\end{tabular}}
    \end{subfigure}
    \caption{Average runtimes of \texttt{strassen} and \texttt{standard} for matrix sizes $n$}
\end{figure}
We observe that, for $n > 30$, \texttt{strassen} is faster than \texttt{standard}. Thus, the empirical crossover point is $n_0 = 30$. 
\\

We also observe that, for $n < 11$, \texttt{standard} is faster than \texttt{strassen}, and for $11 \leq n \leq 30$, it is unclear which algorithm is faster.
\section{Discussion}
\subsection{Difficulties in Implementation}
An interesting difficulty we encountered was how to efficiently implement padding into the algorithm. The initially intuitive solution we have is to pre-process the matrix with padding, and then post-process the result after \texttt{Strassen} algorithm had finished running to remove the pads. However, we wanted yet struggled to find an implementation that incorporated all the padding into the single \texttt{Strassen} function. The various calls of recursion made it difficult to keep track of when un-padding should be done, as defining it within the function would cause issues with recursive calls. \\

It was not necessarily unexpected, but timing the algorithms took a significant number of hours when iterating on the larger matrix sizes. This made it important to manage when edits would be made as to not interfere with the code while the program was running as then we would have to re-run to re-time it with the modifications.  \\

\subsection{Matrix Choice}


We chose to multiply matrices with powers of 2 because, with our implementation of \texttt{Strassen} with inputs of matrix size not of powers of 2, it would require an extra padding operation that would complicate and draw out the time.    
\end{document}
